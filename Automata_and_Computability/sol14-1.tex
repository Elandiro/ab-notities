\documentclass[a4paper]{article}
\usepackage[dutch]{babel}
\usepackage{color}
\usepackage{../assignment-nl,../brackets,../importsreferences-nl,tikz}

\title{Automaten en Berekenbaarheid\\Oplossingen \#1\\\url{http://goo.gl/1RlVG5}}
\author{prof. B. Demoen\\W. Van Onsem}
\date{Oktober 2014}

\topmargin -15mm 
\textwidth 16truecm 
\textheight 24truecm
\oddsidemargin 5mm 
\evensidemargin 5mm


\newcommand{\R}{\mathcal{R}}

\begin{document}

\maketitle

\begin{question}
Voor een string $w = w_1 w_2 \ldots w_n$ duiden we de omgekeerde string $w_n \ldots w_2  w_1$ aan met $w^{\R}$ en voor een taal $L$ schrijven we $L^{\R} = \condset{w^{\R}}{w\in L}$. Als $L$ regulier is, is $L^{\R}$ dan ook regulier?
\begin{answer}
Ja. We kunnen dit bewijzen op twee manieren: (1) door een constructie voor te stellen die voor elke reguliere expressie $e$, een reguliere expressie $e^{\R}$ genereert;  (2) een constructie die gegeven een NFA $n$ een NFA $n^{\R}$ genereert.
\begin{construction}
Gegeven een NFA $n=\tupl{Q,\Sigma,\delta,q_s,F}$. We beschouwen een toestand $q'$ die niet in de originele toestandsverzameling $Q$. We beschouwen ook een nieuwe transitiefunctie $\delta'$ zodat:
\begin{equation}
\forall q_i\in Q,\sigma\in\Sigma_\epsilon:\fun{\delta'}{q_i,\sigma}=\condset{q_j\in Q}{q_i\in\fun{\delta}{q_j,\sigma}}
\end{equation}
en
\begin{equation}
\fun{\delta'}{q',\epsilon}=F
\end{equation}
De geconstrueerde NFA is dan: $n^{\R}=\tupl{Q\cup\accl{q'},\Sigma,\delta',\accl{q'},q_s}$
\end{construction}
\end{answer}
\end{question}

\begin{question}
Geef voor elk van de volgende talen over het alfabet $\accl{0,1}$ een reguliere expressie die die taal bepaalt. Geef ook een NFA die die taal aanvaardt.
\begin{enumerate}
  \item $\condset{w}{\mbox{op elke oneven positie van $w$ staat een $1$}}$
  \item $\condset{w}{\mbox{$w$ bevat minstens twee $0$'en en hoogstens \'e\'en $1$}}$
  \item Het complement van $\accl{11,111}$
  \item $\condset{w}{\mbox{$w$ bevat evenveel maal de substring $01$ als de substring $10$}}$
\end{enumerate}
\begin{answer}~~
\begin{enumerate}
  \item $\brak{1\brak{0|1}}^{\star}\brak{1|\epsilon}$
  \item $0^{\star}\brak{00|001|010|100}0^{\star}$
  \item $\epsilon|0\brak{0|1}^{\star}|10\brak{0|1}^{\star}|110\brak{0|1}^{\star}|111\brak{0|1}\brak{0|1}^{\star}$
  \item $0^{\star}|1^{\star}|0\brak{0|1}^{\star}0|1\brak{0|1}^{\star}1$
\end{enumerate}
\begin{figure}
\importtikzsubfigure{exc1-dfa1}{}
\importtikzsubfigure{exc1-dfa2}{}
\importtikzsubfigure{exc1-dfa3}{}
\importtikzsubfigure{exc1-dfa4}{}
\caption{NFA's voor de gegeven talen}
\end{figure}
\end{answer}
\end{question}

\begin{question}
Toon aan dat er voor elke $n \geq 1$ NFA's bestaan die de volgende talen aanvaarden.
\begin{enumerate}
  \item $\condset{a^k}{\mbox{$k$ is een veelvoud van $n$}}$ over het alfabet $\accl{a}$.
  \item $\condset{x}{\mbox{$x$ is de binaire voorstelling van een natuurlijk getal dat een veelvoud is van $n$}}$
\end{enumerate}
\begin{answer}~
\begin{enumerate}
 \item Men kan een DFA construeren die deze taal beslist al volgt:
 \begin{construction}
  Men beschouwt $n$ toestanden die we nummeren $q_1,q_2,\ldots,q_n$. $s_n$ is de starttoestand en de enige accepterende toestand. De transitiefunctie $\delta$ is gedefinieerd als volgt:
  \begin{equation}
   \forall q_i\in Q, j\in\accl{0,1}:\fun{\delta_1}{q_i,a}=\acclguard{q_{i+1}&\xif{i<n}\\q_{1}&\xotherwise}
  \end{equation}
  De volledige NFA is dan $\tupl{\accl{q_1,q_2,\ldots,q_n},\accl{a},\delta_1,q_n,\accl{q_n}}$.
 \end{construction}
 \item We beschouwen een little-endian automaat: de minst significante bits bevinden zich op het einde van de string. Merk op dat dit geen noodzakelijke vereiste is: we weten immers door vraag~1 dat we door de omgekeerde taal te beschouwen, een big-endian automaat kunnen construeren. Men kan een DFA construeren die deze taal beslist al volgt:
 \begin{construction}
  Men beschouwt $n$ toestanden die we nummeren $q_0,q_1,\ldots,q_{n-1}$. $s_0$ is de starttoestand en de enige accepterende toestand. De transitiefunctie $\delta$ is gedefinieerd als volgt:
  \begin{equation}
   \forall q_i\in Q, j\in\accl{0,1}:\fun{\delta_2}{q_i,j}=q_k\xwhere k=2\cdot i+j\mod n
  \end{equation}
  De volledige NFA is dan $\tupl{\accl{q_0,q_1,\ldots,q_{n-1}},\accl{0,1},\delta_2,q_0,\accl{q_0}}$.
 \end{construction}
\end{enumerate}
\end{answer}
\end{question}

\end{document}
