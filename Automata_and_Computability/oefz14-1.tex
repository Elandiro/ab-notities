\documentclass[a4paper]{article}
\usepackage[dutch]{babel}
\usepackage{color}
\usepackage{../assignment-nl,../brackets}

\title{Automaten en Berekenbaarheid\\Opgave \#1\\\url{http://goo.gl/1RlVG5}}
\author{prof. B. Demoen\\W. Van Onsem}
\date{Oktober 2014}


\newcommand{\R}{\mathcal{R}}

\begin{document}

\maketitle

\begin{question}
Voor een string $w = w_1 w_2 \ldots w_n$ duiden we de omgekeerde string $w_n \ldots w_2  w_1$ aan met $w^{\R}$ en voor een taal $L$ schrijven we $L^{\R} = \condset{w^{\R}}{w\in L}$. Als $L$ regulier is, is $L^{\R}$ dan ook regulier?
\end{question}

\begin{question}
Geef voor elk van de volgende talen over het alfabet $\accl{0,1}$ een reguliere expressie die die taal bepaalt. Geef ook een NFA die die taal aanvaardt.
\begin{enumerate}
  \item $\condset{w}{\mbox{op elke oneven positie van $w$ staat een $1$}}$
  \item $\condset{w}{\mbox{$w$ bevat minstens twee $0$'en en hoogstens \'e\'en $1$}}$
  \item Het complement van $\accl{11,111}$
  \item $\condset{w}{\mbox{$w$ bevat evenveel maal de substring $01$ als de substring $10$}}$
\end{enumerate}
\end{question}

\begin{question}
Toon aan dat er voor elke $n \geq 1$ NFA's bestaan die de volgende talen aanvaarden.
\begin{enumerate}
  \item $\condset{a^k}{\mbox{$k$ is een veelvoud van $n$}}$ over het alfabet $\accl{a}$.
  \item $\condset{x}{\mbox{$x$ is de binaire voorstelling van een natuurlijk getal dat een veelvoud is van $n$}}$
\end{enumerate}
\end{question}

\end{document}
