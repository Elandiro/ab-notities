\documentclass[a4paper]{article}
\usepackage[dutch]{babel}
\usepackage{color}

\topmargin -15mm 
\textwidth 16truecm 
\textheight 24truecm
\oddsidemargin 5mm 
\evensidemargin 5mm

\newcommand{\kolom}[1]{ \left[ \begin{array}{c} #1 \end{array} \right] }



% Opmerkingen na oefenzitting:
%	- Opgave 1 en 3 staan in de cursus (eventueel vervangen door oude examenvragen)

\begin{document}
\section*{Oefenzitting 2}
	\begin{enumerate}
		\item Als $L$ een reguliere taal is over een alfabet $\Sigma$, dan is de taal $L^c = \{ w \ | \ \mbox{$w$  is een string over $\Sigma$ en $w \not\in L$} \}$ ook regulier. Bewijs.   % Sipser 1.10 (a) zonder de hint
		\item Zij $\Sigma_3$ het alfabet 
				{\tiny
					\[ \left\{ \kolom{ 0 \\ 0 \\ 0}, \kolom{ 0 \\ 0 \\ 1}, \kolom{ 0 \\ 1 \\ 0}, \ldots, \kolom{ 1 \\ 1 \\ 1} \right\}. \]
				}
				Beschouw een string over $\Sigma_3$ als drie rijen van $0$'en en $1$'en, en beschouw elk van die rijen als de binaire representatie van een natuurlijk getal. Zij $L$ de taal 
				\[ \{ w \ | \ \mbox{de onderste rij van $w$ is de som van de twee bovenste rijen} \} \]
				over $\Sigma_3$. {\tiny $\kolom{ 0 \\ 0 \\ 1} \kolom{ 1 \\ 0 \\ 0} \kolom{ 1 \\ 1 \\ 0}$} behoort bijvoorbeeld tot $L$, maar { \tiny $\kolom{ 0 \\0 \\ 1} \kolom{ 1 \\ 0 \\ 1}$} niet. Toon aan dat $L$ regulier is.
		\item Toon aan dat als $L_1$ en $L_2$ reguliere talen zijn, $L_1 \cap L_2$ dan ook een reguliere taal is. Is $L_1 \setminus L_2$ ook regulier?
		\item Een \emph{all-paths}-NFA $(Q,\Sigma,\delta,q_0,F)$ verschilt van een NFA doordat een string $w$ enkel aanvaard wordt als 
				\begin{itemize}
					\item voor elke mogelijke schrijfwijze $w = y_1 y_2 \ldots y_m$, $y_i \in \Sigma_{\varepsilon}$ en elke sequentie $r_0, r_1, \ldots, r_m$ van toestanden zodanig dat $r_0 = q_0$ en $r_{i+1} \in \delta(r_i,y_{i+1})$, geldt dat $r_m \in F$;
               \item er minstens \'e\'en schrijfwijze $w = y_1 y_2 \ldots y_m$, $y_i \in \Sigma_{\varepsilon}$ bestaat zodanig dat er een sequentie $r_0, \ldots, r_m$ van toestanden bestaat met $r_0 = q_0$ en $r_{i+1} \in \delta(r_i,y_{i+1})$.
            \end{itemize}
				Toon aan dat een taal $L$ regulier is als en slechts als ze herkend wordt door een all-paths-NFA.
		\item Noem een string $x$ een \emph{prefix} van een string $y$ als er een string $z$ bestaat zodanig dat $y = xz$. Als bovendien $x \neq y$, dan noemen we $x$ een \emph{echte prefix} van $y$. Toon aan dat de klasse van reguliere talen gesloten is onder de volgende operaties.
				\begin{enumerate}
					\item $\mathrm{NOPREFIX} (L) = \{ w \in L \ | \ \mbox{geen enkele echte prefix van $w$ zit in $L$} \}$
					\item $\mathrm{NOEXTEND} (L) = \{ w \in L \ | \ \mbox{$w$ is geen echte prefix van een string in $L$} \}$
				\end{enumerate}
%		\item Zij $x$ en $y$ twee strings over een alfabet $\Sigma$ en $L$ een taal over $\Sigma$. We zeggen dat $x$ en $y$ van elkaar te onderscheiden zijn door $L$ als er een string $z$ bestaat zodanig dat precies \'e\'en van de strings $xz$ en $yz$ tot $L$ behoort. Als voor elke string $z$ geldt dat $xz \in L$ als en slechts als $yz \in L$, dan zeggen ze dat $x$ en $y$ niet te onderscheiden zijn door $L$. Dit noteren we met $x \equiv_L y$.
%				\begin{enumerate}
%					\item Maak een DFA $M$ over een alfabet $\Sigma$ en twee strings $x$ en $y$ over $\Sigma$ zodanig dat $x$ en $y$ van elkaar te onderscheiden zijn door $L_M$.                                  % Niet uit Sipser
%					\item Toon aan dat $\equiv_L$ een equivalentierelatie is.    % Sipser 1.34
%				\end{enumerate}
%				Zij $X$ een verzameling strings. We zeggen dat $X$ paarsgewijs te onderscheiden is door $L$ als elke twee strings uit $X$ van elkaar te onderscheiden zijn door $L$. We noemen het maximum aantal elementen dat zo een paarsgewijs te onderscheiden verzameling kan bevatten de \emph{index van $L$}.
%				\begin{enumerate}
%					\setcounter{enumii}{2}
%					\item Toon aan dat als $L$ herkend wordt door een DFA met $k$ toestanden, de index van $L$ dan hoogstens $k$ is.          % Sipser 1.35 (a)
%					\item Toon aan dat als de index van $L$ een eindig getal $k$ is, $L$ herkend kan worden door een DFA met $k$ toestanden.  % Sipser 1.35 (b)
%				\end{enumerate}
	\end{enumerate}
\end{document}
