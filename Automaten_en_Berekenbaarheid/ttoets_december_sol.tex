\documentclass[fleqn]{article}
\title{Automaten en Berekenbaarheid:\\opgave 2}
\author{Prof. B. Demoen (\url{Bart.Demoen@cs.kuleuven.be})\\ W. Van Onsem (\url{Willem.VanOnsem@cs.kuleuven.be})}
\date{12 december 2013}
\usepackage{tikz,../assignment-nl,amssymb}
\usetikzlibrary{shapes}
\newcommand{\lang}[1]{\textsc{#1}}
\newcommand{\cons}{\mbox{CONS}}
\newcommand{\NN}{\ensuremath{\mathbb{N}}}

\tikzstyle{every picture}+=[remember picture]
\everymath{\displaystyle}
\tikzstyle{na} = [baseline=-.5ex]

\begin{document}
\maketitle
\paragraph{Richtlijnen}

Je schrijft enkel op de gekleurde bladen die je gekregen hebt: op je
werkplek liggen geen andere bladen dan die gekleurde bladen en de
bladen van de cursustekst. De bladen van je cursustekst waarop
oplossingen van oefeningen of aanwijzingen tot oplossingen van
oefeningen staan, steek je ook weg. Je jas, je boekentas, je gsm
... leg je vooraan in het auditorium, zoals gebruikelijk bij een
examen. Bij het afgeven zorg je dat je naam op je bladen staat, en dat
je alles aan elkaar niet. Er zijn 3 vragen.

\begin{question}[Herkenbaar, beslisbaar, regulier, context-vrij,...?]
Waar of niet waar? Beargumenteer/bewijs.
\begin{itemize}
 \item $\mbox{EQ}_{\mbox{\small{CFG}}}=\accl{\tupl{G_1,G_2}| \mbox{Twee context-vrije grammatica's $G_1$ en $G_2$ met $\fun{L}{G_1}=\fun{L}{G_2}$}}$ is co-herkenbaar.
 \item Bepalen of er voor een gegeven Java-programma een invoer bestaat zodat de Java-emulator een gemarkeerde lijn code zal uitvoeren is beslisbaar.
 \item $\mbox{A}_{\mbox{\small{CSG}}}=\accl{\tupl{G,x}| \mbox{Een context-sensitieve grammatica $G$ en een string $s\in\Sigma^*$ met $s\in\fun{L}{G}$}}$ is herkenbaar.
 \item $\accl{\tupl{M}|M\mbox{ is een Turingmachine waarbij de taal $\fun{L}{M}$ een eindig aantal strings bevat}}$ is beslisbaar, herkenbaar, en/of co-herkenbaar.
 \item $\accl{\tupl{M}|M\mbox{ is een Turingmachine waarbij de taal $\fun{L}{M}$ minder dan $k$ strings bevat}}$ met $k$ een vaste parameter is beslisbaar, herkenbaar, en/of co-herkenbaar.
\end{itemize}
\end{question}
\begin{answer}\hfill
\begin{itemize}
 \item \textit{Waar}. Een PDA kan beslissen (in eindige tijd dus) of een
gegeven string door een gegeven grammatica bepaald wordt.  Een
Turing machine $M$ die $EQ_{CFG}$ co-herkent werkt dus als volgt: bij de
twee gegeven grammatica's wordt de overeenkomstige PDA geconstrueerd;
die PDA's worden vervolgens gesimuleerd op een opeenvolging van
strings (in lexicografische orde bijvoorbeeld), \'{e}\'{e}n string per
keer, en op het moment dat een string gevonden wordt waarop de PDA's een
verschillend antwoord geven, accepteert $M$.

 \item \textit{Niet waar}. Je kan de taal $\tupl{\mbox{Javaprogramma},\mbox{gemarkeerde lijn}}$
{\em Turing-reduceren} naar een taal $\tupl{TM,\mbox{set toestanden}}$ met de vraag of
er een invoer voor de $TM$ bestaat die de uitvoering in \'e\'en van de toestanden
brengt die overeen komt met de gemarkeerde lijn. Daarvan hebben we in de oefeningen gezien dat het niet
beslisbaar is.


 \item \textit{Waar}. Dit komt neer op het parsen van een string voor
een context sensitieve grammatica. Het acceptance probleem voor
context sensitieve grammatica's is beslisbaar als volgt: zet de CSG om
in een LBA\footnote{Dit deden we niet in de cursus, maar is wel
vermeld als eigenschap van CSG}, en voeg (zoals in de cursus)
lus-detectie toe aan die LBA. Bepaald worden door een CSG is dus
equivalent met $A_{LBA}$ hetgeen beslisbaar is.


 \item \textit{Noch herkenbaar noch co-herkenbaar}. De eigenschap {\em
$L_M$ bevat een eindig aantal strings} is een niet-triviale,
taal-invariante eigenschap van Turing machines, dus volgens Rice is die
eigenschap niet beslisbaar. Bijgevolg is de verzameling van TMs met
een eindige taal niet herkenbaar of niet co-herkenbaar (misschien
beide).

Je kan $A_{TM}$ (Turing-)reduceren naar $\mbox{FINITE}_{TM}$ en naar
$\mbox{INFINITE}_{TM}$ als volgt:
\begin{enumerate}
\item $\tupl{M,w}\rightarrow M_{1w}$: op input $s$ laat $M$ $\abs{s}$ stappen lopen op $w$,
   	     	     indien $M$ ondertussen accept, dan accepteert $M_{1w}$ $s$
		     	                          anders reject $M_{1w}$ $s$

\item $\tupl{M,w}\rightarrow M_{2w}$: op input s laat die $M$ $\abs{s}$ stappen lopen op $w$,
                     indien $M$ ondertussen accept, dan reject $M_{2w}$ $s$
		     	                          anders accept $M_{2w}$ $s$
\end{enumerate}
$M$ accepteert $w$ als en slechts als $M_{1w}$ heeft oneindige taal
$M$ accepteert $w$ als en slechts als $M_{2w}$ heeft eindige taal

we hebben dus een reductie van een niet herkenbare taal
$\overline{A_{TM}}$ naar $FINITE_{TM}$ en $INFINITE_{TM}$ dus $\mbox{FINITE}_{TM}$ is niet herkenbaar, en niet co-herkenbaar.



\item \textit{Co-herkenbaar}. We gebruiken Rice om te laten zien dat de taal niet beslisbaar is:
de taalinvariante, niet-triviale eigenschap is {\em $\abs{L_M} \leq k$ }

Hier is een TM die gegeven een herkent dat $\abs{L_M} > k$:
maak van $M$ een enumerator die elke herkende string slechts 1 keer op
de outputband zet; vanaf dat er meer dan $k$ strings op de tape staan
accept. Indien $\abs{L_M} > k$ stopt dit, anders niet altijd.
\end{itemize}
\end{answer}
\begin{question}[Lambda calculus]\brak{f\ x}
\begin{itemize}
 \item Ontwerp een zuivere $\lambda$-expressie hier $\mbox{il}$ genoemd zodat voor elke $i\geq 0$ geldt: $\fun{\mbox{head}}{\fun{\mbox{tail}^i}{\mbox{il}}}=\fun{f^i}{x}$. Mocht men dus de expressie $\mbox{il}$ evalueren zou dit resulteren in volgende oneindige expressie:
 \begin{equation}
  \fun{\cons}{x,\fun{\cons}{ \fun{f}{x}, \fun{\cons}{ \fun{f}{\fun{f}{x}}, \fun{\cons}{ \fun{f}{\fun{f}{\fun{f}{x}}} , \ldots } } }}
 \end{equation}
 Evalueer ook de expressie \funm{head}{\funm{tail}{\mbox{il}}} volgens reductie in normaalorde.
 \item In lambda-calculus zijn variabelen niet altijd gebonden. Omcirkel bij onderstaande expressie de gebonden variabelen en wijs met pijl de overeenkomstige lambda aan. Bijvoorbeeld:
 \[
 \tikz[baseline]{\node(l){$\lambda$};}\ \tikz[baseline]{\node(y){x};}\ \tikz[baseline]{\node(x){.};}\ \tikz[baseline]{\node[draw,circle](x){x};}
 \]
\begin{tikzpicture}[overlay]
\path[->] (x) edge [bend left] (l);
\end{tikzpicture}
 Ongebonden variabelen duid je aan met een hoedje ($\hat{x}$).
 \begin{equation}
  +\ y\ \lambdacal{x}{+\ y\ \lambdacal{y}{\lambdacal{x}{+\ \brak{+\ y\ y}\ x}\ y}}\lambdacal{y}{+\ y\ x}
 \end{equation}
\end{itemize}
\end{question}
\begin{answer}\hfill
\begin{itemize}
 \item We defini\"eren een helperfunctie $\mbox{il2}$ die we tijdens de uitwerking vaak zullen gebruiken:
 \begin{equation}
  \mbox{il2} = Y\ \lambdacal{g\ f\ x}{\cons\ x \brak{g\ f\ \brak{f\ x}}}
 \end{equation}
 . De expressie voor $\mbox{il}$ is dan de volgende:
\begin{equation}
\mbox{il}=\mbox{il2}\ f\ x = Y\ \lambdacal{g\ f\ x}{\cons\ x \brak{g\ f\ \brak{f\ x}}}\ f\ x
\end{equation} ofwel voluit
\begin{equation}
\lambdacal{g}{\lambdacal{x}{g\ x\ x}\lambdacal{x}{g\ x\ x}}\ \lambdacal{g\ f\ x}{\lambdacal{a\ b\ f}{f\ a\ b}\ x \brak{g\ f\ \brak{f\ x}}}\ f\ x
\end{equation}
De werking kan ge\"illustreerd worden met behulp van de \mbox{FAC} functie: de $Y$ operator wordt vervangen door $Y\ \mbox{ily}\rightarrow g\ \brak{Y\ \mbox{ily}}$ met $\mbox{ily}$ de $\mbox{il2}$ functie zonder $Y$ operator. Dit betekent dus dat $g$ in het gedeelte rechts van $Y$ zal binden met $\mbox{il2}$. De functie vereist verder twee parameters $f$ en $x$, als resultaat geven wen een $\mbox{CONS}$-structuur terug, wat dus een head-tail element is van een lijst. De head wordt bepaald door de parameter $x$, de tail rekenen we uit door $\mbox{il2}$ ofwel $g$ verder te propageren. We roepen deze $\mbox{il2}$ op met dezelfde functie $f$ maar met $f\ x$ in plaats van $x$. Doordat deze oproep $\mbox{il2}$ op zijn beurt weer verder zal propageren, ontstaat de gewenste sequentie.\\
We zullen de evaluatie met behulp van de compacte notatie uitvoeren en enkel operatoren ontbinden wanneer dit nodig is:
\begin{eqnarray}
\mbox{head}\ \brak{\mbox{tail}\ \brak{\mbox{il2}\ f\ x}}\hfill\\
\lambdacal{c}{c\ \lambdacal{a\ b}{a}}\ \brak{\mbox{tail}\ \brak{ \mbox{il2}\ f\ x }}\hfill\\
\brak{\mbox{tail}\ \brak{ \mbox{il2}\ f\ x }}\ \lambdacal{a\ b}{a}\hfill\\
\brak{\lambdacal{c}{c\ \lambdacal{a\ b}{b}}\ \brak{ \mbox{il2}\ f\ x }}\ \lambdacal{a\ b}{a}\hfill\\
\brak{\brak{ \mbox{il2}\ f\ x } \lambdacal{a\ b}{b}}\ \lambdacal{a\ b}{a}\hfill\\
\brak{\brak{ \brak{ Y\ \lambdacal{g\ f\ x}{\cons\ x \brak{g\ f\ \brak{f\ x}}} }\ f\ x } \lambdacal{a\ b}{b}}\ \lambdacal{a\ b}{a}\hfill\\
\brak{\brak{ \lambdacal{f\ x}{\cons\ x \brak{\mbox{il2}\ f\ \brak{f\ x}}}\ f\ x } \lambdacal{a\ b}{b}}\ \lambdacal{a\ b}{a}\hfill\\
\brak{\cons\ x \brak{\mbox{il2}\ f\ \brak{f\ x}} \lambdacal{a\ b}{b}}\ \lambdacal{a\ b}{a}\hfill\\
\brak{\lambdacal{h}{h\ x \brak{\mbox{il2}\ f\ \brak{f\ x}}} \lambdacal{a\ b}{b}}\ \lambdacal{a\ b}{a}\hfill\\
\brak{\lambdacal{a\ b}{b}\ x \brak{\mbox{il2}\ f\ \brak{f\ x}} }\ \lambdacal{a\ b}{a}\hfill\\
\brak{\brak{ Y\ \lambdacal{g\ f\ x}{\cons\ x \brak{g\ f\ \brak{f\ x}}} }\ f\ \brak{f\ x} }\lambdacal{a\ b}{a}\hfill\\
\brak{\lambdacal{f\ x}{\cons\ x \brak{\mbox{il2}\ f\ \brak{f\ x}}}\ f\ \brak{f\ x} }\lambdacal{a\ b}{a}\hfill\\
\cons\ \brak{f\ x} \brak{\mbox{il2}\ f\ \brak{f\ \brak{f\ x}}}\lambdacal{a\ b}{a}\hfill\\
\lambdacal{h}{h\ \brak{f\ x} \brak{\mbox{il2}\ f\ \brak{f\ \brak{f\ x}}}}\lambdacal{a\ b}{a}\hfill\\
\lambdacal{a\ b}{a}\ \brak{f\ x} \brak{\mbox{il2}\ f\ \brak{f\ \brak{f\ x}}}\hfill\\
f\ x\hfill
\end{eqnarray}
\item Zie onderstaande expressie: \[
\tikz[baseline]{\node(p1){+};}\ \enctikz{temp}{\hat{y}}\ \lambdacaltikz{x1}{x}{\enctikz{tmp}{\enctikz{tmp}{+\ \hat{y}}}\ \lambdacaltikz{y1}{y}{\lambdacaltikz{x2}{x}{\enctikz{tmp}{+}\ \enctikz{tmp}{(+}\ \enctikzd{y1l1}{y}\ \enctikzd{y1l2}{y}\enctikz{tmp}{)}\ \enctikzd{x1l1}{x}}\ \enctikzd{y1l3}{y}}}\lambdacaltikz{y2}{y}{\enctikz{tmp}{+}\ \enctikzd{y2l1}{y}\ \enctikz{tmp}{\hat{x}}}
 \]
\begin{tikzpicture}[overlay]
\foreach \x/\y/\m in {y1l1/y1/left,y1l2/y1/right,x1l1/x2/left,y1l3/y1/right,y2l1/y2/left} {
  \path[->] (\x) edge [bend \m] (la\y);
}
\end{tikzpicture}
\end{itemize}
\end{answer}
\begin{question}[Primitief recursief en \textsc{for}-programma's]
Primitief recursieve programma's zijn even sterk als \textsc{for}-programma's: programma's die als enig controle-mechanisme een \verb+for+-lus kunnen defini\"eren. De grenzen (bijvoorbeeld \verb+m+ en \verb+n+) moet voor het binnengaan van de for-lus gekend zijn en zowel de teller als de grens mogen niet aangepast worden tijdens uitvoer van de lus. Recursie is uiteraard ook verboden (anders zou men via recursie for-lussen met aanpasbare teller/grenzen kunnen emuleren). Een geldig \textsc{for}-programma is bijvoorbeeld:
\begin{verbatim}
function f(m,n)
    for i = m:n
        m = m*n;
    endfor
    for i = 0:m
        n = n+i;
    endfor
    return n;
endfunction
\end{verbatim}
Een ongeldig \textsc{for}-programma is bijvoorbeeld:
\begin{verbatim}
function f(m,n)
    for i = m:n
        i = i*m;
    endfor
    for i = 0:m
        m = m-i;
        i = 2*i;
        n = f(m+1,n-1);
    endfor
    return n;
endfunction
\end{verbatim}
Beschrijf aan de hand van een reeks transformatieregels hoe men een gegeven primitief-recursieve functie kan omzetten naar een \textsc{for}-programma.
\end{question}
\begin{answer}
De basisfuncties zoals $\mbox{nul}$, $\mbox{suc}$ en $\mbox{p}_i^n$ hebben een straight-forward implementatie:
\begin{verbatim}
function nul(n)
    return 0;
endfunction

function succ(n)
    return n+1;
endfunction
\end{verbatim}
De $i$-de projectie uit $n$ elementen:
\begin{verbatim}
function pin(x1,x2,...,xn)
    return xi;
endfunction
\end{verbatim}
De compositie $\mbox{Cn}\left[f,g_1,g_2,\ldots,g_m\right]$ met $f:\NN^m\rightarrow\NN$ en voor elke $g_i$, $g_i:\NN^k\rightarrow\NN$ wordt dan als volgt gedefinieerd:
\begin{verbatim}
function cn(x1,x2,...,xk)
    y1 = g1(x1,x2,...,xk);
    y2 = g2(x1,x2,...,xk);
    //...
    ym = gm(x1,x2,...,xk);
    return f(y1,y2,...,ym);
endfunction
\end{verbatim}
Tot slot de primitieve recursie functie $\mbox{Pr}\left[f,g\right]$ met $f:\NN^k\rightarrow\NN$ en $g:\NN^{k+2}\rightarrow\NN$:
\begin{verbatim}
function pr(x1,x2,...,xk,l)
    y = f(x1,x2,...,xk);
    for i=1:l
        y = g(x1,x2,...,xk,i,y);
    endfor
    return y;
endfunction
\end{verbatim}
\end{answer}
Indien er in een programma meerdere primitief recursieve en/of compositie-functies voorkomen, moet men natuurlijk verschillende namen defini\"eren.
\paragraph{Voorbeeld}
Bij wijze van voorbeeld: de $\mbox{som}$-functie is gedefinieerd als: $\mbox{Pr}\left[p^1_1,\mbox{Cn}\left[\mbox{succ},p^3_3\right]\right]$. We bekomen dan volgende code:
\begin{verbatim}
function succ(n)
    return n+1;
endfunction

function p11(x1)
    return x1;
endfunction

function p33(x1,x2,x3)
    return x3;
endfunction

function cn(x1,x2,x3)
    y1 = p33(x1,x2,x3);
    return succ(y1);
endfunction

function pr(x1,x2)
    y = p11(x1);
    for i = 1:x2
      y = cn(x1,i,y);
    endfor
    return y;
endfunction

\end{verbatim}
\end{document}