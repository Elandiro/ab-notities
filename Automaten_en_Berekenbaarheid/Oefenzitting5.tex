\documentclass[a4paper]{article}
\usepackage[dutch]{babel}
\usepackage{amsmath}

\newcommand{\rul}{\rightarrow}
\newcommand{\gvar}[1]{\langle \text{{#1}} \rangle}
\newcommand{\gend}[1]{\text{\bf{#1}}}

\topmargin -15mm 
\textwidth 16truecm 
\textheight 24truecm
\oddsidemargin 5mm 
\evensidemargin 5mm


\begin{document}
\pagestyle{empty}

\section*{Oefenzitting 5}

\begin{enumerate}
   \item Welke van de volgende talen zijn context-vrij, welke niet? Bewijs je antwoord.
   % Belangrijk bleek een mooi uitgewerkt voorbeeld in het begin van de oefenzitting te geven.
      \begin{enumerate}
			\item $\{ w \in \{0,1\}^* \ | \ \text{$w$ is een palindroom}\}$
         \item $\{ 0^n\#0^{2n}\#0^{3n} \ | \ n \geq 0 \}$
         % Goed van moeilijkheid 
         \item $\{ w \in \{a,b,c\}^* \ | \ \text{$w$ bevat een gelijk aantal $a$'s, $b$'s en $c$'s} \}$
         % Probeer hiervoor een oefening te vinden zodat het resultaat uit oefening 1 echt gebruikt moet worden. Met andere woorden, een taal die niet rechtstreeks met het pumping lemma aangepakt kan worden.
         \item $\{ xy \ | \ \text{$x,y \in \{0,1\}^*$ en $|x| = |y|$, maar $x \neq y$} \}$
         \item $\{ w \# x \ | \ \text{$w,x \in \{0,1\}^*$ en $w$ is een substring van $x$} \}$
			\item $\{a^ib^jc^k \ | \ 0 \leq i \leq j \leq k \}$
      \end{enumerate}
	\item Een $k$-PDA is een PDA met $k$ stacks. Een $0$-PDA is dus een NFA en een $1$-PDA een gewone PDA. 
   \begin{enumerate}
      \item Toon aan dat $2$-PDA's strikt sterker zijn dan $1$-PDA's. (Met andere woorden, toon aan dat de klasse van talen die met een $1$-PDA te herkennen zijn een strikte deelverzameling is van de klasse van talen die met een $2$-PDA te herkennen zijn.)
      % Gaat redelijk goed vanaf de studenten weten dat ze gewoon een tegenvoorbeeld moeten zoeken.
      \item Toon aan dat $3$-PDA's even sterk zijn als $2$-PDA's.
      % Al wat moeilijker. Formeel opschrijven is voor zowat iedereen niet haalbaar.
   \end{enumerate}

   \item Zij 
   \begin{align*}
      \Sigma & = \{ \gend{if, condition, then, else, a:=1} \} \\
      V & = \{ \gvar{STMT}, \gvar{IF-THEN}, \gvar{IF-THEN-ELSE}, \gvar{ASSIGN} \} \\
   \end{align*}
      en zij $G = (V,\Sigma,R,\gvar{STMT})$ de grammatica met regels
      \begin{align*}                                                                               
         \gvar{STMT}          & \rul \gvar{ASSIGN} \ | \ \gvar{IF-THEN} \ | \ \gvar{IF-THEN-ELSE}  \\
         \gvar{IF-THEN}       & \rul \gend{if condition then } \gvar{STMT} \\
         \gvar{IF-THEN-ELSE}  & \rul \gend{if condition then } \gvar{STMT} \gend{ else } \gvar{STMT} \\
         \gvar{ASSIGN}        & \rul \gend{a:=1} \\
      \end{align*}
      Toon aan dat $G$ ambigu is en construeer een niet-ambigue grammatica die dezelfde taal herkent. 
      % Goed, maar te moeilijk, zelfs voor goede studenten.
\end{enumerate}
\end{document}
