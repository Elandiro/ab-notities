\documentclass[a4paper]{article}
\usepackage[dutch]{babel}
\usepackage{color}

\topmargin -15mm 
\textwidth 16truecm 
\textheight 24truecm
\oddsidemargin 5mm 
\evensidemargin 5mm


\newcommand{\R}{\mathcal{R}}

\begin{document}

\section*{Oefenzitting 1}

\begin{enumerate}
   \item Voor een string $w = w_1 w_2 \ldots w_n$ duiden we de omgekeerde string $w_n \ldots w_2  w_1$ aan met $w^{\R}$ en voor een taal $L$ schrijven we $L^{\R} = \{ w^{\R} \ | \ w \in L\}$.
         Als $L$ regulier is, is $L^{\R}$ dan ook regulier?
	\item Geef voor elk van de volgende talen over het alfabet $\{ 0,1 \}$ een reguliere expressie die die taal bepaalt. Geef ook een NFA die die taal aanvaardt.
			\begin{enumerate}
				\item $\{ w \ | $ op elke oneven positie van w staat een $1 \}$                           % Sipser 1.13 (i)
				\item $\{ w \ | \ w$ bevat minstens twee $0$'en en hoogstens \'e\'en $1 \}$               % Sipser 1.13 (j)
				\item Het complement van $\{ 11,111\}$																		% Sipser 1.13 (h)
				\item $\{ w \ | \ w$ bevat evenveel maal de substring $01$ als de substring $10\}$		   % Sipser 1.41
			\end{enumerate}
	\item	Toon aan dat er voor elke $n \geq 1$ NFA's bestaan die de volgende talen aanvaarden.
			\begin{enumerate}
				\item $\{ a^k \ | \ k $ is een veelvoud van $n \}$ over het alfabet $\{ a \}$.                                 % Sipser 1.29
				\item $\{ x \ | \ x $ is de binaire voorstelling van een natuurlijk getal dat een veelvoud is van $n \}$ \\    % Sipser 1.30
			\end{enumerate}
\end{enumerate}

\end{document}
