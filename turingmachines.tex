\documentclass[main.tex]{subfiles}
\begin{document}

\chapter{Turingmachines en beslisbaarheid}
\label{cha:turingmachines-en-beslisbaarheid}


\begin{de}
  Een \term{Turingmachine} is een $7$-tal $(Q,\Sigma,\Gamma, \delta, q_{s}, q_{a}, q_{r})$:
  \begin{itemize}
  \item $Q$ is een eindige verzameling toestanden
  \item $\Sigma$ is een eindig invoeralfabet dat $\#$ niet bevat.
  \item $\Gamma$ is een eindig bandalfabet dat $\#$ bevat, alsook $\Sigma$.
  \item $q_{s}$ is de starttoestand.
  \item $q_{a}$ is de aanvaardende toestand.
  \item $q_{r}$ is de verwerpende toestand.
  \item $\delta$ is de totale overgangsfunctie.
  \[ \delta:\ Q \times \Gamma \rightarrow Q \times \Gamma \times \{L,R,S\} \]
  \end{itemize}
  Een Turingmachine beschikt over een strook geheugen die aan twee kanten onbegrensd is.
  Deze strook is ingedeeld in vakjes die precies \'e\'en symbool bevatten.
  De Turingmachine beschikt bovendien over een lees- en schrijfeenheid die zich over een vakje bevindt dat de machine bijgevolg kan lezen en waarnaar de machine kan schrijven.
  De Turingmachine bevindt zich op elk moment in precies \'e\'en toestand.
\end{de}

\begin{de}
  De \term{werking van een Turingmachine} definieren we aan de hand van een Turingmachine $M = (Q,\Sigma,\Gamma, \delta, q_{s}, q_{a}, q_{r})$ en een invoerstring $s$.
  \begin{itemize}
  \item \textbf{Initialisatie}\\
    De geheugenstrook wordt eerst ge\"initialiseerd zodat ze enkel het symbool $\#$ bevat in op elk vakje.
    Vervolgens wordt vanaf een willekeurig vakje opeenvolgend de invoerstring in de vakjes op de band gezet.
    De lees- en schrijfeenheid wordt op het vakje gezet waar de string begint.
    De machine bevindt zich aan het begin in de begintoestand $q_{s}$.
  \item \textbf{Uitvoering}\\
    De Turingmachine $M$ voert achtereenvolgens de volgende stappen uit.
    \begin{itemize}
    \item De machine leest het symbool dat onder de lees- en schrijfeenheid staat.
    \item Op basis van de staat $q$ waarin de machine zich bevindt en het symbool $a$ onder de eenheid beslist de machine wat hij zal doen aan de hand van de overgangsfunctie.
      \[ \delta(q,a) = (q',a',r) \]
      De machine zal het symbool $a'$ schrijven, naar staat $q'$ overgaan en zicht in de richting $r$ verplaatsen.
    \end{itemize}
  \item \textbf{Beslissing}\\
    Zodra de machine zich ofwel in $q_{a}$ ofwel in $q_{r}$ bevindt stopt hij.
    Als de machine zich op het einde in $q_{a}$ bevindt zeggen we dat $M$ de string $s$ aanvaardt.
    Als de machine zich op het einde in $q_{r}$ bevindt zeggen we dat $M$ de string $s$ verwerpt.
    Merk op dat het ook kan voorvallen dat de machine bij een gegeven string $s$ niet ophoudt met werken.
  \end{itemize}
\end{de}

\begin{de}
  Elke turingmachine $M$ bepaalt een verdeling van $\Sigma^{*}$ in drie stukken:
  \begin{itemize}
  \item De strings die door $M$ worden geaccepteerd: $L_{M}$.
  \item De strings waarvoor $M$ in een oneindige lus belandt: $\infty_{M}$.
  \item De strings die door $M$ worden verworpen.
  \end{itemize}
\end{de}

\begin{de}
  Een turingmachine $M$ herkent $L_{M}$.
\end{de}

\begin{de}
  Een taal $L$ is \term{Turing-herkenbaar} als er een Turingmachine $M$ bestaat die $L$ herkent.
\end{de}

\begin{de}
  Een turingmachine $M$ beslist een taal $L$ als $M$ $L$ herkent en $\infty_{M}$ leeg is.
\end{de}

\begin{de}
  Een taal $L$ is \term{Turing-beslisbaar} als er een Turingmachine $M$ bestaat die $L$ beslist.
\end{de}

\begin{de}
  Een taal $L$ is \term{co-herkenbaar} als $\bar{L}$ herkenbaar is.
\end{de}

\begin{de}
  Een taal $L$ is \term{co-beslisbaar} als $\bar{L}$ beslisbaar is.
\end{de}

\begin{st}
  Als een taal $L$ zowel herkenbaar als co-herkenbaar is, dan is $L$ beslisbaar (en co-beslisbaar).
  \begin{proof}
    Kies een willekeurige herkenbare en co-herkenbare taal $L$.
    Er bestaat dan een Turingmachine $m$ die $L$ herkent en een Turingmachine $m'$ die $\bar{L}$ herkent.
    Construeer vanuit $m$ en $m'$ nu een machine $M$ die $L$ beslist.
    $M$ bestaat erin $m$ en $m'$ simultaan te laten lopen op de invoerstring $s$
    Zodra $m$ $s$ aanvaardt, aanvaardt $m$ $s$ en zodra $m'$ $s$ aanvaardt, verwerpt $m$ $s$.
    Voor elke mogelijke invoerstring stopt minstens \'en\'en van de machines $m$ en $m'$.
    $M$ zal dus altijd stoppen.
    \clarify{Dit kan formeler}
    \question{Hoe maken we dit formeler?}
  \end{proof}
\end{st}

\begin{st}
  Er bestaat een taal die niet herkenbaar is.
  \begin{proof}
    Het aantal Turingmachines is aftelbaar oneindig (, maar enkel op isomorfisme na).
    \clarify{waarom?}
    Elke Turingmachine herkent bovendien precies \'e\'en taal.
    Het aantal talen over een gegeven alfabet is overaftelbaar oneindig.\footnote{Zie stelling \ref{st:overaftelbaar-veel-talen}.}
    Er bestaat dus minstens \'e\'en taal waarvoor er geen Turingmachine bestaat die die taal herkent.
    Die taal is niet herkenbaar.
    Sterker nog: ``De meeste\footnote{``De meeste'' heeft een heel specifieke wiskundige betekenis.} talen zijn niet herkenbaar.''
  \end{proof}
\end{st}

\begin{de}
  Zij $M = (Q,\Sigma,\Gamma, \delta, q_{s}, q_{a}, q_{r})$ een Turingmachine die zich in een staat $q$ bevindt tijdens de uitvoering. 
  Zij $\alpha$ de niet-$\#$ symbolen links van de schrijfeenheid en $\beta$ de niet-$\#$ symbolen onder en rechts van de schrijfeenheid.
  We noteren dan deze \term{configuratie} van $M$ als volgt.
  \[ \alpha q \beta \]
\end{de}

\end{document}
