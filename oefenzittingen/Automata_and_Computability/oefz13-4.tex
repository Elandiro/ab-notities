\documentclass[a4paper]{article}
\usepackage[dutch]{babel}
\usepackage{amsmath}

\topmargin -15mm 
\textwidth 16truecm 
\textheight 24truecm
\oddsidemargin 5mm 
\evensidemargin 5mm


\begin{document}
\pagestyle{empty}

\section*{Oefenzitting 4}

% Deze oefenzitting is iets te lang.

\begin{enumerate}
\item Een \emph{`blijf staan Turing machine'} verschilt van een gewone Turing machine doordat de head naar rechts kan bewegen of kan blijven staan, maar niet naar links kan bewegen. Toon aan dat deze variant van Turing machines niet equivalent is met de originele variant. Welke talen worden herkend door `blijf staan Turing machines'?
      % Dat het reguliere talen zijn zien de meesten snel in. Niemand vond een redelijk bewijs.
% \item Een \emph{`schrijf eenmaal Turing machine'} is een Turing machine met \'e\'en tape en zodanig dat de inhoud van elk vakje van die tape slechts eenmaal veranderd kan worden. Toon aan dat deze variant van Turing machines equivalent is met de originele variant.
\item Zij $INF_{\rm DFA} = \{ \langle A \rangle \ | \ \text{$A$ is een DFA en $L(A)$ is een oneindige taal} \} $. Toon aan dat $INF_{\rm DFA}$ beslisbaar is.
\item Is $VIJF_{\rm DFA} = \{ \langle A \rangle \ | \ \text{$A$ is een DFA en $L(A)$ bestaat uit precies 5 strings} \} $ beslisbaar? Bewijs je antwoord.
\item Toon aan dat de vraag of een context-vrije grammatica minstens \'e\'en string uit $1^*$ kan genereren, beslisbaar is.
\item Zij $A$ en $B$ twee disjuncte talen die co-Turing-herkenbaar zijn. Toon aan dat er een beslisbare taal $C$ bestaat zodanig dat $A \subseteq C$ en $B \subseteq \overline{C}$. 
%\item Toon aan dat de vraag of een context-vrije grammatica alle strings uit $1^*$ kan genereren eveneens beslisbaar is.
\item Een \emph{Turing machine met RESET} verschilt van een gewone Turing machine doordat de head enkel naar rechts kan bewegen of in \'e\'en stap helemaal aan het begin van de tape gezet kan worden. De transitiefunctie van zo een machine is dus van de vorm $\delta : Q \times \Gamma \to Q \times \Gamma \times \{ \text{R}, \text{RESET} \}$. Toon aan dat deze variant van Turing machines equivalent is met de originele variant.
      % Moeilijk.
\item Een \emph{nutteloze toestand van een TM $M$} is een toestand waarin $M$ voor geen enkele invoer kan komen. Zij $U = \{ \langle M, q \rangle \ | \ \text{$M$ is een TM en $q$ een nutteloze toestand van $M$} \}$. Toon aan dat $U$ onbeslisbaar is.
   % Eenvoudige oefening. Misschien iets naar voren schuiven in deze oefenzitting.
\item Toon aan dat $R_{\rm DFA} = \{ \langle A \rangle \ | \ \text{$A$ is een DFA die $\widehat{w}$ aanvaardt asa hij $w$ aanvaardt} \}$ beslisbaar is. Is $R_{\rm TM} = \{ \langle M \rangle \ | \ \text{$M$ is een TM die $w$ aanvaardt asa ze $\widehat{w}$ aanvaardt} \}$ ook beslisbaar? Bewijs je antwoord.
   % Goed van niveau. De studenten die mee zijn met de leerstof kunnen deze oefening aan.
\item Toon aan dat de taal van alle Turing machines $M$ waarvoor er een invoer bestaat zodanig dat $M$ bij die invoer een niet-blanco symbool zal overschrijven met een blanco symbool, onbeslisbaar is.
\end{enumerate}

\end{document}
