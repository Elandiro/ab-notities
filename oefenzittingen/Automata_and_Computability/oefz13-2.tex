\documentclass[a4paper]{article}
\usepackage[dutch]{babel}
\usepackage{color}
\usepackage{xypic}
\xyoption{all}

\topmargin -15mm 
\textwidth 16truecm 
\textheight 24truecm
\oddsidemargin 5mm 
\evensidemargin 5mm

\newcommand{\kolom}[1]{ \left[ \begin{array}{c} #1 \end{array} \right] }



% Opmerkingen na oefenzitting:
%	- Opgave 1 en 3 staan in de cursus (eventueel vervangen door oude examenvragen)

\begin{document}
\section*{Oefenzitting 2}
	\begin{enumerate}
		\item Als $L$ een reguliere taal is over een alfabet $\Sigma$, dan is de taal $L^c = \{ w \ | \ \mbox{$w$  is een string over $\Sigma$ en $w \not\in L$} \}$ ook regulier. Bewijs.   % Sipser 1.10 (a) zonder de hint
		\item Zij $\Sigma_3$ het alfabet 
				{\tiny
					\[ \left\{ \kolom{ 0 \\ 0 \\ 0}, \kolom{ 0 \\ 0 \\ 1}, \kolom{ 0 \\ 1 \\ 0}, \ldots, \kolom{ 1 \\ 1 \\ 1} \right\}. \]
				}
				Beschouw een string over $\Sigma_3$ als drie rijen van $0$'en en $1$'en, en beschouw elk van die rijen als de binaire representatie van een natuurlijk getal. Zij $L$ de taal 
				\[ \{ w \ | \ \mbox{de onderste rij van $w$ is de som van de twee bovenste rijen} \} \]
				over $\Sigma_3$. {\tiny $\kolom{ 0 \\ 0 \\ 1} \kolom{ 1 \\ 0 \\ 0} \kolom{ 1 \\ 1 \\ 0}$} behoort bijvoorbeeld tot $L$, maar { \tiny $\kolom{ 0 \\0 \\ 1} \kolom{ 1 \\ 0 \\ 1}$} niet. Toon aan dat $L$ regulier is.
		\item Toon aan dat als $L_1$ en $L_2$ reguliere talen zijn, $L_1 \cap L_2$ dan ook een reguliere taal is. Is $L_1 \setminus L_2$ ook regulier?
		\item Een \emph{all-paths}-NFA $(Q,\Sigma,\delta,q_0,F)$ verschilt van een NFA doordat een string $w$ enkel aanvaard wordt als 
				\begin{itemize}
					\item voor elke mogelijke schrijfwijze $w = y_1 y_2 \ldots y_m$, $y_i \in \Sigma_{\varepsilon}$ en elke sequentie $r_0, r_1, \ldots, r_m$ van toestanden zodanig dat $r_0 = q_0$ en $r_{i+1} \in \delta(r_i,y_{i+1})$, geldt dat $r_m \in F$;
               \item er minstens \'e\'en schrijfwijze $w = y_1 y_2 \ldots y_m$, $y_i \in \Sigma_{\varepsilon}$ bestaat zodanig dat er een sequentie $r_0, \ldots, r_m$ van toestanden bestaat met $r_0 = q_0$ en $r_{i+1} \in \delta(r_i,y_{i+1})$.
            \end{itemize}
				Toon aan dat een taal $L$ regulier is als en slechts als ze herkend wordt door een all-paths-NFA.
		\item Noem een string $x$ een \emph{prefix} van een string $y$ als er een string $z$ bestaat zodanig dat $y = xz$. Als bovendien $x \neq y$, dan noemen we $x$ een \emph{echte prefix} van $y$. Toon aan dat de klasse van reguliere talen gesloten is onder de volgende operaties.
				\begin{enumerate}
					\item $\mathrm{NOPREFIX} (L) = \{ w \in L \ | \ \mbox{geen enkele echte prefix van $w$ zit in $L$} \}$
					\item $\mathrm{NOEXTEND} (L) = \{ w \in L \ | \ \mbox{$w$ is geen echte prefix van een string in $L$} \}$
				\end{enumerate}
				\item Minimaliseer de volgende DFA \\
          \[  \entrymodifiers={++[o][F-]}
            \SelectTips{cm}{}
            \xymatrix {
              *{} \ar[r] &
              1               \ar[r]_0    \ar[rd]^1                              &
              *++[o][F=]{2}   \ar[d]_1    \ar `ur_r[dd]  `[dd] `_dl[dd]^0 [dd]     &
              3               \ar `ur_d[dd] `_l[dd]^1 [dd]   \ar[d]_0                               \\
              *{} &
              4               \ar@(lu,ld)[]_1   \ar@/_/[r]_0      &
              5               \ar[d]_1          \ar@/_/[l]_0      &
              *++[o][F=]{6}   \ar@(ru,rd)[]_1   \ar[d]_0          \\
              *{} &
              7               \ar[u]_1          \ar[ru]_0                     &
              *++[o][F=]{8}   \ar@(ld,rd)[]^0   \ar@/_/[r]_1                  &
             9                \ar@/_/[l]_1      \ar `dr_l[ll] `_ur[ll]^0 [ll]  \\
         } \]

   \item Zij $\Sigma_1 = \{0,1\}, \Sigma_2 = \{a\}$ en $\Sigma_3 = \{a,b,c\}$. Toon aan dat de volgende talen niet regulier zijn.
      \begin{enumerate}
         \item $\{ w \in \Sigma_1^*\ |\ \mbox{$\left|w\right|$ is even en de $1$'en in $w$ komen enkel voor in de laatste helft van $w$}\}$
         \item $\{ a^{\left( 2^n \right) } \in \Sigma_2^* \ | \ n \geq 0 \}$                                      % Sipser 1.17 (c)
         \item $\{ a^n \in \Sigma_2^*\ | \ \mbox{$n$ is een priemgetal} \}$                                       % Niet uit Sipser
         \item $\{ 0^m1^n \in \Sigma_1^* \ | \ m \neq n \}$                                                       % Sipser 1.23 (c)
         \item $\{ w \in \Sigma_1^* \ | \ \mbox{$w$ is geen palindroom} \}$                                       % Sipser 1.23 (d)
         \item $\{ a^ib^jc^k \in \Sigma_3^* \ | \ \mbox{$i,j,k \geq 0$ en als $i = 1$, dan moet $j=k$} \}$        % Sipser 1.37
      \end{enumerate}
	\end{enumerate}
\end{document}
