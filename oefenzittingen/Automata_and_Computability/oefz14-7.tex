\documentclass[a4paper]{article}
\usepackage[dutch]{babel}
\usepackage{color,xypic,amsmath}
\xyoption{all}
\usepackage{../assignment-nl,../brackets}

\title{Automaten en Berekenbaarheid\\Opgave \#7\\\url{http://goo.gl/1RlVG5}}
\author{prof. B. Demoen\\W. Van Onsem}
\date{November 2014}


\begin{document}
\maketitle

\begin{question}
Een \emph{nutteloze toestand van een TM $M$} is een toestand waarin $M$ voor geen enkele invoer kan komen. Zij $U = \{ \langle M, q \rangle \ | \ \text{$M$ is een TM en $q$ een nutteloze toestand van $M$} \}$. Toon aan dat $U$ onbeslisbaar is.
\end{question}

\begin{question}
Toon aan dat $R_{\rm DFA} = \{ \langle A \rangle \ | \ \text{$A$ is een DFA die $\widehat{w}$ aanvaardt asa hij $w$ aanvaardt} \}$ beslisbaar is. Is $R_{\rm TM} = \{ \langle M \rangle \ | \ \text{$M$ is een TM die $w$ aanvaardt asa ze $\widehat{w}$ aanvaardt} \}$ ook beslisbaar? Bewijs je antwoord.
\end{question}

\begin{question}
Toon aan dat de taal van alle Turing machines $M$ waarvoor er een invoer bestaat zodanig dat $M$ bij die invoer een niet-blanco symbool zal overschrijven met een blanco symbool, onbeslisbaar is.
\end{question}

\begin{question}
Toon aan dat het `overeenkomstprobleem van Post' beslisbaar is over het alfabet $\Sigma = \{ a \}$. (Dat wil zeggen: toon aan dat het voor een verzameling domino's van de vorm $\left[ \frac{a^m}{a^n} \right]$ met $n,m \in \mathbb{N}$, beslisbaar is of er een overeenkomst bestaat.) 
\end{question}

\begin{question}
Toon aan dat er een onbeslisbare deelverzameling van $1^*$ bestaat.
\end{question}

\begin{question}
Toon aan dat de vraag of een context-vrije grammatica $G$ alle strings uit $1^*$ kan genereren beslisbaar is.
\end{question}

\end{document}
