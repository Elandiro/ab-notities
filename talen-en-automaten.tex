\documentclass[main.tex]{subfiles}
\begin{document}

\chapter{Talen en Automaten}
\label{cha:talen-en-automaten}

\section{Symbolen en Strings}
\label{sec:symbolen-en-strings}

\begin{de}
  Een \emph{symbool} $s$ is een representatie van een object in de abstractste zin van het woord. 
\end{de}

\begin{de}
  Een alfabet $\Sigma$ is een verzameling van symbolen.
\end{de}

\begin{de}
  Een \emph{string} $s$ over een alfabet $\Sigma$ is een geordende opeenvolging van nul, \'e\'en of   meer elementen van $\Sigma$.
\end{de}

\begin{de}
  $\epsilon$ is de string zonder symbolen en noemen we de \emph{lege string}.
\end{de}

\section{Talen}
\label{sec:talen}

\begin{de}
  Een \emph{taal} $L$ over een alfabet $\Sigma$ is een verzameling van eindige strings over $\Sigma$.
\end{de}

\begin{de}
  De \emph{concatenatie} $xy$ van twee strings $x = \{x_1,x_2,\ldots,x_m\}$ en $y =   \{y_1,y_2,\ldots,y_n\}$ is de volgende geordende verzameling  
  \[
  xy = \{ x_1,x_2,\ldots,x_m,y_1,y_2,\ldots,y_n\}
  \] 
\end{de}

\begin{ei}
  De concatenatie van strings is associatief:
  \[
  (xy)z = x(yz)
  \]
  \begin{proof}
    \[
    \begin{array}{r l l}
      (xy)z &= \{x_1,x_2,\ldots,x_m,y_1,y_2,\ldots,y_n\}z &\\
            &= \{x_1,x_2,\ldots,x_m,y_1,y_2,\ldots,y_n,z_1,z_2,\ldots,z_o\} &\\
            &= x\{y_1,y_2,\ldots,y_n,z_1,z_2,\ldots,z_o\} &= x(yz)
    \end{array}
    \]
  \end{proof}
\end{ei}

\begin{de}
  De \emph{concatenatie} $L_1L_2$ \emph{van twee talen} $L_1$ en $L_2$ over hetzelfde alfabet $\Sigma$ is de volgende verzameling:
  \[
  L_1L_2 = \{\ xy\ |\ x \in L_1,\ y \in L_2\ \} 
  \]
\end{de}

\begin{ei}
  De concatenatie van talen is associatief:
  \[
  (L_1L_2)L_3 = L_1(L_2L_3)
  \]
  \begin{proof}
    \[
    \begin{array}{r l l}
      (L_1L_2)L_3 &= \{\ xy\ |\ x \in L_1,\ y \in L_2\ \}L_3 &\\
                  &= \{\ xyz\ |\ x \in L_1,\ y \in L_2\,\ z \in L_3\} &\\
                  &= L_1\{\ yz\ |\ y \in L_2,\ z \in L_3\ \} &= L_1(L_2L_3)
    \end{array}
    \]
  \end{proof}
\end{ei}

\begin{de}
  De concatenatie van $n$ keer een taal $L$ met zichzelf noteren we als $L^n$.
  $L^0$ bevat enkel de lege string.
  \[
  L^0 = \{\epsilon\}
  \]
\end{de}

\begin{de}
  De \emph{Kleene ster} $L^*$ van een taal $L$ is de unie van alle concatenaties van $L$ met zichzelf.
  \[
  L^* = \bigcup_{n=0}^{\infty}L^n
  \]
\end{de}

\begin{de}
  $L^{+}$ is de unie van $L$, \'e\'en of meer keer geconcateneerd met zichzelf.
\end{de}

\section{Reguliere expressies en talen}
\label{sec:reguliere-expressies-en-talen}

\begin{de}
  Een \emph{reguliere expressie} (RE) wordt inductief gedefinieerd als een expressie van de volgende vorm:
  \begin{itemize}
  \item $\epsilon$
  \item $\phi$
  \item $a$ met $a \in \Sigma$
  \item $(E_1E_2)$ waarbij $E_1$ en $E_2$ reguliere expressies zijn over $\Sigma$
  \item $(E)^*$ waarbij $E$ een reguliere expressie is over $\Sigma$
  \item $(E_1|E_2)$ waarbij $E_1$ en $E_2$ reguliere expressies zijn over $\Sigma$
  \end{itemize}
\end{de}

\begin{de}
  De \emph{taal $L_E$ bepaald door een reguliere expressie} over hetzelfde alfabet $\Sigma$ is de volgende.
  \[
  \begin{array}{|c|c|}
    \hline
    E & L_E\\
    \hline
    a \text{ met } a \in \Sigma & {a}\\
    \epsilon & {\epsilon}\\
    \phi & \emptyset\\
    (E_1E_2) & L_{E1}L_{E2}\\
    (E) & L_E^*\\
    (E_1|E_2) & L_{E1} \cup L_{E2}\\
    \hline
  \end{array}
  \]
\end{de}

\begin{de}
  Een \emph{reguliere taal} is een taal die bepaald wordt daar een reguliere expressie.
\end{de}

\begin{de}
  $RegLan$ is \emph{verzameling van alle reguliere talen}.
\end{de}

\begin{de}
  $L_{\Sigma}$ is \emph{de verzameling van alle talen over $\Sigma$}. 
\end{de}

\end{document}
